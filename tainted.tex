\section{Tainted types to the rescue}
\label{sec:tainted}
%
In this section, we introduce several design patterns that are employed the 
\sys sandboxing APIs.
%
These design patterns center around the use of \tainted---a simple wrapper 
types that allow the \sys APIs to ensure that untrusted data cannot 
be misused by application developers without too much additional burden or 
modification of application code.
%
Furthermore, they also allow incremental migration of the application code to 
use sandboxed libraries. 
%
We discuss the role of these design patterns in the \sys APIs in more detail 
below; we also generalize these design patterns and show how wrapper types may 
be leveraged to solve challenges in related problem domains.

\para{Design pattern 1: Wrap untrusted data in \tainted type}
%
This means that when calling a function in the sandboxed library that 
returns in \code{int}, this must return a \taintedW{int} instead.
%
Similarly when sandboxed code invokes a permitted callback in the application 
code, all parameters to the callback must be \tainted.
%
Wrapping such untrusted data in \tainted types, imposes a monadic interface on 
this data which allows us to prevent potentially unsafe operations such as 
performing an array access with untrusted indexes.
%
For instance, consider the code below which invokes to sandboxed functions from 
a host application.
%
\begin{minted}[linenos=false, breaklines=true, mathescape=true,
escapeinside=||]{cpp}
int numbers[] = {24, 48, 72};
int i = sandboxed_get_index();
// We want a compiler error below
int num = numbers[i]; 
sandboxed_set_value(num);
\end{minted}
%
It is important to disallow this pattern: since \code{i} is untrusted, 
indexing into the array \code{numbers} may result in an out of bounds 
read operation which is then passed back to the untrusted code forming an 
arbitrary read gadget.
%
The \sys API does by providing an API called \code{invoke_sbx_func} tha 
takes the function name as a parameter and automatically wraps the return value 
in a \tainted type.
%
Thus, the above code would hit a compilation error since the type of \code{i} is
\taintedW{int}.
%
This same approach can also prevent other unsafe operations such as branching 
on \tainted values, \tainted comparisons and more.
%
Instead of directly allowing use, \tainted data must explicitly require a 
validation function be used as below
%
\begin{minted}[linenos=false, breaklines=true, mathescape=true,
escapeinside=||]{cpp}
int i_checked = i.verify([](int val){
  runtime_assert(i >= 0 && i <= 2);
});
\end{minted}
%
Here, the compiler cannot check the correctness of the validation function, but 
developers would at least get an error if they have used the value without any 
checking.
%
This design pattern ensures that application developers do not use untrusted 
data with validation by accident.
%
While some prior sandboxing APIs~\cite{codejail, google-sandboxed-api} do wrap 
some of te untrusted data, most SFI toolchain APIs do not do this. 
%
Even toolchains that wrap untrusted data do not always do so consistently (they 
wrap data returned from untrusted functions, but not the parameters of 
callbacks).

However wrapping untrusted data is a much more general pattern and can be 
applied in other well known security boundaries.
%
For instance, operating system kernel code frequently handle userspace 
pointers, but must be careful to never dereference them before checking. 
%
In fact, prior work~\cite{cqual-kernel-ptr} has proposed using C's attributes 
feature to achieve the same affect of having a wrapped type.
%
Another example are applications running on the trusted execution environments 
such as Intel SGX~\tocite{sgx} must frequently interface with untrusted code; 
indeed code from the host OS is also untrusted in this context.
%
In fact, when reviewing a variety of TEE runtimes, Van Bulck et 
al.~\cite{two-worlds-sgx} encounter a that almost all frameworks have gubs 
pertaining to use of unchecked data.
%
The problems discussed in this SGX context, have a substantial overlap with the 
problems we independently discovered later trying to safely used sandboxed 
libraries when we challenge of using sandboxed libraries that indeed prompted 
the need for the \sys API. 
%
Wrapping untrusted data is a simple but powerful idea and can largely reduce 
this type of bugs by leveraging the compiler checking in static type systems.

However, this still does not reduce the burden on application developers to 
write validation checks themselves---the next two design patterns focus on this.


\Red{
  Design pattern 2: Construct APIs around tainted types.
  This can be apply to all user input apis in a lang std library 
  such as scanf, getline etc. This would certainly put some types of user 
  validation bugs in more sharp focus - sql injection, path traversal 
  bugs, regex DOS attacks. Languages and libraries could also safely 
  provide APIs that consume tainted data in place of requiring validators 
  --- a tainted regex should be processed by the poly time algorithm, 
  rather than one with an exponential worst case

}

\para{Design pattern 2: Construct APIs with \tainted types}
%
Consider the validator required for the following piece of code.
%
\begin{minted}[linenos=false, breaklines=true, mathescape=true,
escapeinside=||]{cpp}
tainted<char*> src = invoke_sbx_func(sandbox_get_src);
tainted<char*> dst = invoke_sbx_func(sandbox_get_dst);
unsigned int num = ...;

memcpy(dst.verify(??), src.verify(??), num);
\end{minted}
%
Here the application code retrieves the location of some source and destination 
byte buffer located in sandbox memory and must transfer a certain number of 
bytes between them.
%
To use \code{memcpy}, the application developer must apply some validation to 
the \code{dst} and \code{src} buffers.
%
In this case, the validation check is a more subtle than it may appear---to 
ensure that \code{memcpy} does not inadvertently read or write application 
memory, the validation check must ensure that both \code{dst + num} and 
\code{src + num} is inside sandbox memory \emph{and} that the addition 
operation has not overflowed.
%
Rather than requiring application developers to 

\Red{

Design pattern 3: Use a separate \vtainted type for memory under 
attacker control. Allows running checkers for TOCTOU and double fetch. 
Can also leverage things like Intel transactional memory to ``freeze'' 
the memory. This also provides an clear avenue for ABI 
conversions---\vtainted is always represented in the ABI of the 
sandboxed component while \tainted uses the application ABI. This 
automatic ABI conversion is super useful---SFI compilers have different 
ABIs. Additionally consider use of other security lowfat pointers, or 
even hardware features CHERI. It may not be possible to migrate your 
entire application to use this without breaking compatibility, or 
impacting perf. However, an automatic ABI conversion would allow you to 
apply this to selectively to higher risk components and having \tainted 
types perform the ABI conversions on boundary crossings.

Design pattern 4: Allow benign operations. Tainted computations. This 
allows the user to retain the syntax of the language --- you don't have 
to have custom constructs or have to unwrap data before doing anything 
with data. More importantly allows pushing validators further into 
code. This in turn allows converting 
checking of library invariants to application invariants, which are 
much easier to enforce. Give example of PNG. This additionally allows 
automatic lazy marshaling of data and lazy ABI conversions (see point 
2). Outside of sandboxing, this is extremely useful in general RPC 
scenarios as well---you can potentially improve performance be avoiding 
unserializing function returns when they are not needed.

Design pattern 5: Automate common security checks on tainted types. 
Some domains allow automated checks. FOr example dereferencing a 
tainted pointer during sandboxing.

Design pattern 6: Provide a clearly marked escape hatch that removed 
the tainted wrapper. This is key to incremental porting. In fact in the 
sandboxing case, we provide an additional escape hatch as well called 
the null sandbox. This is necessary so that ABI differences don't 
affect incremental porting, but additionally and allows a 
compile/runtime setting to turn off sandboxing depending on 
requirement. This also allows for downstream sandboxing.

Optional design pattern 7: Only allow tainted pointers as parameters to 
sandboxed function calls. This prevents complicated marshalling 
requirements for nested data structures going in. Same thing for 
callback returns.
}